\chapter*{Conclusion}
\addcontentsline{toc}{chapter}{Conclusion}
I thus participated in a project to help the research in biology and in medicine.
The interactions between proteins can indedd allow the synthetisation of new medicine.
 As it matters, these interactions can be understood by analyzing the interface of
  contact between both proteins of a complex. The goal of the project was to model
   this interface in the form of a surface in three dimensions.

   Using a method based on the Delaunay triangulation and the
   Voronoï diagram, we saw that it is possible to approximate this surface.
   To implement this method, we used a C ++ librarywhich helps developping
geometrical algorithms named CGAL. The structures and functions were of a paramount
 importance for the good progress of the project.
     Using the data for the modelling of the surface is indeed facilitated
     by the classes provided by CGAL and the iterators which their are bound to. Going through
Delaunay structures constitutes one of the most important stages of
       the development. After having modelled the interface of contact between two proteins, we
 were capable of visualizing our results thanks to \textit{.off } files
  and MeshLab.

  Finally, the method was a success, in spite of some difficulties in term of compilation.
   On the other hand, the CGAL library proved its utility and its efficiency
   despite its sometimes heavy and laborious use.
  The main objective was fulfilled: the visualization of the interface is functional and
  the structure of the project allows to implement new features.
