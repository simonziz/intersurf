\chapter*{Introduction}
\addcontentsline{toc}{chapter}{Introduction}

The INRIA (Institut National de Recherche en Informatique et en Automaique)
is a public French
institute of research in computer science and mathematics.
Founded in 1967, the INRIA
accounts for 2600 collaborators gathered on various sites in
France, one of which being Nancy. Within this center, the team CAPSID develops
algorithms and softwares allowing to study biological phenomena and systems
from a structural point of view, thanks to 3D modelling. I did my internship
with this team under the supervision of Dave Ritchie who is its manager.

Overseen by Dave Ritchie and Bernard Maigret, this project's main goal is to
model the interface of contact between two proteins. The properties of the interface
can indeed give a lot of information about the interactions between
proteins. This is particularly heplful in fields such as biology and medicinal search for
the development of new medicine.
The project also takes place in partnership with the research team Vegas, and
especially with Olivier Devillers, who participated in the implementation of CGAL
(Computational Geometry Algorithms Library).

This library allows, through to the numerous features it offers,
to improve and to accelerate the development of the method chosen for the project.
We will give more details in this report on the theoretical method adopted
to approximate the interface between proteins: the Delaunay triangulation and the
Voronoï Diagram. We will also explain how the structures supplied by CGAL are used during the
development of the software by insisting on some of the most crucial parts
of the implementation.
Finally, we will see how the results (and the difficulties encountered)
allow to elaborate future prospects for this project.
