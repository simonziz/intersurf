\chapter*{Conclusion}
\addcontentsline{toc}{chapter}{Conclusion}

J'ai donc participé à un projet visant à aider la recherche en biologie et en médecine.
En effet, les interactions entre les protéines peuvent permettre la synthétisation de
nouveaux médicaments. Or, ces interactions peuvent être comprises en analysant l'interface
de contact entre les deux protéines d'un complexe. L'objectif du projet était de modéliser
cette interface sous forme d'une surface en trois dimensions.

En utilisant une méthode basée sur la triangulation de Delaunay et le diagramme de
Voronoï, nous avons vu qu'il est possible d'approximer cette surface.
Pour implémenter cette méthode, nous nous sommes appuyés sur une bibliothèque C++ qui
aide à développer des algorithmes liés à la géométrie, du nom de \gls{cgal}. Les structures et fonctions fournies
ont été d'une importance fondamentale au bon déroulement du projet. En effet, l'utilisation
des données utiles à la modélisation de la surface est grandement facilité par les classes de \gls{cgal}
et les itérateurs qui leurs sont liés. Le parcours des structures de Delaunay constitue
l'une des étapes les plus importantes du développement.
Après avoir modélisé l'interface de contact entre deux protéines, nous avons été capables
de visualiser nos résultats grâce aux fichiers \textit{.off} et au logiciel MeshLab.

Finalement, la méthode de travail aura été fructueuse, malgré des difficultés rencontrées
en terme de compilation. D'autre part, la bibliothèque \gls{cgal} a prouvé son utilité
et son efficacité en dépit de son utilisation parfois lourde et laborieuse.
L'objectif principal a été rempli : la visualisation de l'interface est fonctionnelle
la structure du projet permet d'envisager l'apport de nouvelles fonctionnalités.
Ainsi, le passage à un affichage en mode dynamique avec OpenGL pourrait être
implémenté.
