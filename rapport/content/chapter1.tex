\chapter{Méthode + Théorie}

\section{Structure d'une protéine et d'un complexe}
\begin{itemize}
  \item Atomes
  \item Interface
  \item .pdb
\end{itemize}
\cite{Proteine}

L'objectif du projet étant de modéliser l'iinterface de contact entre deux protéines,
il est important de comprendre la structure d'une protéine. Nous verrons également
comment sont stockées ces structures, grâce au fichiers \textit{.pdb}.

\subsection{Structure}

Une protéine est composée d'atomes

\subsection{Fichier de stockage}

Les données utilisables d'une proéine (ou d'un complexe) sont stockées grâce aux
fichiers \textit{.pdb}. Ces fichiers, leur lecture et l'interprétation des données
qu'ils contiennent, sont essentiels à l'observation des protéines. En effet, chaque
ligne, exceptée la première, correspond à un atome composant la protéine étudiée.
Ces lignes contiennent des informations telles que la chaîne à laquelle appartient
l'atome, son acide aminé ou ses coordonnées dans l'espace (en $\si{\angstrom}$).   

\begin{figure}[ht]
  \includegraphics[width=\textwidth]{figures/pdb_example.png}
  \caption{Exemple de fichier .pdb}
  \label{fig::pdb_file}
\end{figure}

\section{Triangulation de Delaunay}
\begin{itemize}
  \item Nuage de points
  \item Explication + schema
  \item Dual : Diagramme de Voronoï
\end{itemize}

\begin{figure}[ht]
\centering
\begin{subfigure}{0.4\textwidth}
  \centering
  \includegraphics[width=\textwidth]{figures/delaunay.png}
  \caption{Triangulation de Delaunay}
  \label{fig::delaunay_tr}
\end{subfigure}%
\begin{subfigure}{0.2\textwidth}
  \centering
  $\Longrightarrow$
\end{subfigure}%
\begin{subfigure}{0.4\textwidth}
  \centering
  \includegraphics[width=\textwidth]{figures/delaunay_reduced.png}
  \caption{Triangulation réduite}
  \label{fig:delaunay_reduced}
\end{subfigure}
\caption{A figure with two subfigures}
\label{fig:delaunays}
\end{figure}

\begin{figure}[ht]
\centering
\begin{subfigure}{0.45\textwidth}
  \centering
  \includegraphics[width=\textwidth]{figures/process_d_1.png}
  \caption{Triangulation de Delaunay}
  \label{fig::process_d_1}
\end{subfigure}%
\begin{subfigure}{0.1\textwidth}
  \centering
  $\Longrightarrow$
\end{subfigure}%
\begin{subfigure}{0.45\textwidth}
  \centering
  \includegraphics[width=\textwidth]{figures/process_d_2.png}
  \caption{Arêtes à l'interface}
  \label{fig:process_d_2}
\end{subfigure}
\caption{Triangulations et zone utile}
\label{fig:delaunays_process_1}
\end{figure}

\begin{figure}[ht]
\centering
\begin{subfigure}{0.45\textwidth}
  \centering
  \includegraphics[width=\textwidth]{figures/process_d_3.png}
  \caption{Diagramme de Voronoï}
  \label{fig::process_d_3}
\end{subfigure}%
\begin{subfigure}{0.1\textwidth}
  \centering
  $\Longrightarrow$
\end{subfigure}%
\begin{subfigure}{0.45\textwidth}
  \centering
  \includegraphics[width=\textwidth]{figures/process_d_4.png}
  \caption{Interface}
  \label{fig:process_d_4}
\end{subfigure}
\caption{Recherche de la surface de contact}
\label{fig:delaunays_process_2}
\end{figure}

\begin{figure}[ht]
  \includegraphics[width=\textwidth]{figures/3d_triangulation.png}
  \caption{Triangulation de Delaunay 3D }
  \label{fig::delaunay_3d}
\end{figure}

\section{CGAL}
\begin{itemize}
  \item Structures
  \item Iterateurs
  \item Compilation $\to$ Cmake
\end{itemize}
