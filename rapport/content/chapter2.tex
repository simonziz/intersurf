\chapter{Développement Logiciel}

Dans cette partie, nous expliquerons le développement du programme en différentes étapes :
l'architecture du programme, la récupération des données, la construction de la surface
et l'affichage (voir figure \ref{fig::flow_chart}).

\begin{figure}[ht]
\centering
  \includegraphics[width=\textwidth]{figures/flow_chart.png}
  \caption{Flow chart du programme}
  \label{fig::flow_chart}
\end{figure}

\section{Architecture Logicielle}

Les différentes étapes du programme nécessitent la mise en commun de plusieurs langages
\begin{itemize}
  \item C : lecture du fichier \textit{.pdb}
  \item C++ : traitement des nuages de points avec CGAL
\end{itemize}
De plus, de nombreuses bibliothèques ont été utilisées pour permettre le bon fonctionnement
de \gls{cgal} telles que \textit{Boost} ou \textit{QT5}. Le problème étant que l'installation pouvant
être indépendante de CGAL, les chemins pour la compilation peuvent différer.

La documentation CGAL indique que la compilation s'effectue via un Cmake. Nous avons donc choisi
de compiler le projet entier de cette manière, afin d'obtenir un seul executable.
Cette méthode de compilation a l'avantage de ne pas nécessiter de modification lors de l'ajout
de fichier au projet. Cependant, ayant son propre langage, elle est plus longue à prendre en
main et pose des difficultés lorsque le projet regroupe un grand nombre de bibliothèques
ou de dépendances, ce qui est notre cas.


\section{Récupération de données et stockage}


Le point de départ du projet est de lire un fichier \textit{.pdb} et de le stocker
de manière à créer les structures utilisées par CGAL. Nous avons utilisé une partie du programme
\textbf{Hex} développé en C par David Ritchie qui permet d'analyser le fichier
\textit{.pdb} pour récupérer les données utiles du complexe étudié. Ces informations constituent
la référence utilisée lors de la construction de l'interface lorsque des détails
sur les atomes doivent être connus. Par exemple, si l'on veut connaître la chaîne à
laquelle appartient un atome, il faut remonter jusqu'à ce fichier. La fonction utilisée pour cela
renvoie un pointeur vers un tableau recensant toutes les informations utiles, c'est-à-dire
une image du fichier \textit{.pdb} que nous appellerons \textit{Image pdb} (voir
figure \ref{fig::read_pdb} en Annexe).

Pour créer une structure de Delaunay, CGAL prend normalement une liste de points stockée
sous forme de tableau. Cependant, utiliser cette méthode simple ne premettrait pas d'accéder aux
informations contenues dans le fichier \textit{.pdb}. En effet, chaque sommet de la
triangulation de Delaunay doit être indexé de manière à retrouver les informations voulues dans la
liste d'atomes fournie par le fichier de départ. Les indices doivent donc être choisis
au moment de la lecture de l'\textit{Image pdb} pour correspondre à la place de l'atome
dans le tableau.
Il existe deux méthodes pour insérer un index à une variable de type Vertex dans CGAL :
\begin{itemize}
  \item on modifie la classe Vertex en ajoutant un attribut
  \item on utilise une classe différente fournie par CGAL
\end{itemize}
Nous avons choisi d'utiliser la seconde méthode, qui est native dans \gls{cgal}, pour
sa simplicité. Lors de l'instanciation, cette classe prend notamment en paramètre
\textit{template} un type de variable qui sera accessible comme l'attribut \textit{info()}.

Cette variable constitue l'indice et doit donc être un entier positif qui fera partie de la structure Delaunay
et sera accessible lors du parcours de ces structures. Cependant, la structure de Delaunay
requiert l'ajout d'un indice lors de son instanciation. Un vecteur de paires contenant
un point 3D et une variable du type de l'attribut \textit{info()} convient.

La fonction complète permettant d'écrire un fichier\textit{.pdb} dans une structure
\textit{CGAL::Delaunay} est disponible en annexe (voir figure \ref{fig::read_pdb}).
On remarque la présence de la \textit{map} C++ \textit{is\_interface} qui sera utilisée par la suite
pour sélectionner les tétrahèdres présents à l'interface (voir figure \ref{fig::delaunays}).




\section{Construction de la surface}


Lorsque les strucures de Delaunay ont été créées, il devient possible de calculer l'interface
entre les deux protéines du complexe, ce qui constitue la partie centrale du projet.
En implémentant la méthode vue dans la partie Triangulation de Delaunay, nous pouvons générer
une surface qui pourra être affichée. Cette surface est un regroupement de polyhèdres
dont le nombre de côtés varie.



Dans la première partie de l'implémentation, nous stockons la surface dans \textit{vecteur}
(\textit{all\_faces\_indexes})
et une \textit{map} (\textit{surf\_points})
qui permettent de garder en mémoire les coordonnées des points
et les indices permettant d'afficher les faces. Une \textit{map} est une structure de données
utilisée notamment en C++. Ces données sont stockées sous forme de paires (ou couples)
de variables appelées clef et valeur. Il est alors possible d'accéder à une variable
valeur de la \textit{map} grâce à la clef qui lui correspond, chaque clef étant unique.
Dans notre cas, les clefs seront les points 3D de la surface, avec pour valeur les indices
de ces points.
La \textit{map}  recense tous les points
de la surface une fois, c'est-à-dire que les points utilisés dans différentes faces de la surface
ne sont présents qu'une seule fois dans la map. Cela permet de réduire considérablement
la quantité de données et d'accélerer l'affichage. A chaque point de la map correspond
un indice initialisé à 0 et qui sera mis à jour lors de l'écriture des fichiers utilisés
pour l'affichage. Ensuite, le \textit{vecteur} regroupe toutes les faces de la surface
sous forme de vecteurs de points 3D. Le vecteur \textit{all\_faces\_indexes} contiendra
donc autant de vecteurs que le nombre de faces de la surface.

Nous allons maintenant détailler les différentes parties de la fonction écrite pour
le calcul de l'interface (\textit{calculateInterface()}) qui est disponible en
annexe (voir figure \ref{fig::calculate_interface}).

Dans un premier temps, nous utilisons un itérateur fourni par CGAL pour parcourir
les arêtes de la triangulation. Dans cette itération, on vérifie pour chaque arête
si elle est à l'interface, c'est-à-dire
si elle possède un atome de chaque chaîne. Pour cela, nous nous référons à l'\textit{Image pdb}
grâce à l'indice stocké dans l'attribut \textit{info()} de chaque sommet. Il est possible
d'accéder à la valeur de la chaîne (A ou B) avec la commande en figure \ref{fig::access_chain}.
\begin{figure}[ht]
\centering
  \includegraphics[width=0.6\textwidth]{figures/access_chain.png}
  \caption{Accès à la chaîne à laquelle appartient un sommet}
  \label{fig::access_chain}
\end{figure}
Dans cette commande, \textit{first} correspond à la cellule en cours d'accès et
\textit{second} à l'indice du premier sommet de l'arête (\textit{third} permettra
d'accéder au second sommet de l'arête).

Après cette vérification, nous utilisons un circulateur pour "tourner" autour de l'arête
concernée. En effet, le dual de Voronoï d'une arête dans un espace en trois dimension
est un polyèdre. Celui-ci est composé des points qui sont les duaux des cellules (tétrahèdres)
adjacentes à l'arête étudiée. Le circulateur permet donc de parcourir les cellules adjacentes
des arêtes à l'interface. Chacun de ces points est alors stocké dans le \textit{vecteur} et
la \textit{map} de la manière expliquée plus haut.

Il est important d'expliquer d'autres vérifications qui sont efféctuées avant de stocker
chaque point. Par exemple, il existe dans la structure \textit{Delaunay} de CGAL
un point "infini" permettant de "fermer" la triangulation dans l'espace. Ce point, bien
qu'utile aux calculs effectués par CGAL, ne doit pas être pris en compte car il n'est pas réel.
Nous vérifions donc qu'aucun des tetrahèdres adjacents à l'arête parcourue ne contient
ce point spécifique. De plus, si l'on récupère tous les points générés par le calcul du
diagramme de Voronoï, certains d'entre eux se trouveront en dehors de la zone utile à l'analyse
de l'interface. Pour éliminer ces points, nous avons procédé en amont au calcul
du barycentre de la protéine puis nous avons recensé la distance maximale entre un point du complexe
et son barycentre. Si la distance entre un point d'une face et le barycentre calculé
est supérieure à un certain pourcentage (ici 130\%) de la distance maximale, alors
la face ne sera pas gardée.


La surface est donc stockée dans un \textit{vecteur} et une \textit{map} qui permettront
d'écrire les fichiers \textit{.off} que nous utiliserons pour l'affichage (voir partie
suivante). Ces fichiers ont l'avantage d'être directement lisibles, s'ils sont valides,
grâce une autre classe CGAL (\textit{Polyhedron}) qui permet de stocker
une surface en trois dimension.
Or, la validité de cette surface peut poser problème. En, effet, la classe \textit{Polyhedron}
suppose que les arêtes sont orientées (voir figure \ref{fig::polyheder}). La surface
sera considérée comme invalide si l'orientation des arêtes dans une face change de sens.
\begin{figure}[ht]
\centering
  \includegraphics[width=0.8\textwidth]{figures/polyheder.png}
  \caption{Classe Polyhedron \cite{CGAL}}
  \label{fig::polyheder}
\end{figure}

Pour assurer la validité de la surface, il faut garder le même sens de parcours
tout au long du remplissage de la surface. Nous choisissons pour cela arbitrairement le sens
du circulateur en fonction du premier sommet rencontré (voir figure \ref{fig::choix_sens}).
Nous nous assurons ainsi que le parcours du nuage de points s'effectue toujours dans le même
sens.
\begin{figure}[ht]
\centering
  \includegraphics[width=0.8\textwidth]{figures/choix_sens.png}
  \caption{Choix du sens de circulation}
  \label{fig::choix_sens}
\end{figure}



 L'avantage de cette classe est qu'elle possède
une méthode permettant d'appliquer une subdivision de Catmull-Clark.
Cette subdivision permet de lisser la surface par morceaux
pour l'affichage.


\section{Affichage}

Pour l'affichage des protéines et de l'interface, nous avons choisi les fichiers
\textit{.off} qui permettent de stocker une liste de points (colorés ou non) et
d'indiquer les liens entre chacun de ces points (voir figure \ref{fig::off_file}).

\begin{figure}[ht]
\centering
  \includegraphics[width=0.3\textwidth]{figures/off_file.png}
  \caption{Exemple de fichier \textit{.off}}
  \label{fig::off_file}
\end{figure}

La première ligne indique le type de fichier (OFF) et la présence ou non de coloration :
[C] si oui un espace vide sinon. La seconde ligne donne, dans cet ordre, le nombre
de points, le nombre de cellules (polyhèdres) et le nombre d'arêtes, ce dernier n'étant
pas nécessaire à la lecture du fichier. Dans notre exemple, les lignes 4 à 22 listent
les coordonnées des points et la couleur associée à chacun. La couleur est stockée en RGB
(Rouge, Vert, Bleu) avec des entiers entre 0 et 255 ou des flottants entre 0.0 et 1.0.

Au delà de la ligne 23, les cellules sont listées, avec comme premier entier à chaque
ligne, le nombre de points de la cellule. Comme notre exemple représente une triangulation
de Delaunay, ce nombre vaut 3 car chaque face de la triangulation correspond à un
triangle (cela ne sera plus valable pour la surface dont les faces peuvent comporter un nombre
quelconque de points). A la suite de cet entier viennent les indices des points composant la cellule
(ou la face). Cet indice correspond à la place des coordonnées listées plus haut. Ainsi,
sans notre exemple, le point listé ligne 4 a pour indice 0, celui en ligne 5 a pour indice 1
et ainsi de suite.

La fonction qui permet d'écrire ce fichier depuis la \textit{map} et le \textit{vecteur}
stockant la surface est disponible en annexe (voir figure \ref{fig::save_surface_off}).
Les deux points principaux sont le parcours de la \textit{map} pour écrire les coordonnées
des points grâce à la bibliothèque standard du C++ (\textit{std::ofstream}).
Dans cette boucle, nous mettons à jour l'indice
de chaque point pour lui donner la valeur de sa place dans la liste des points (voir
pseudo code).


\begin{lstlisting}
Initialiser un compteur à 0
Pour tous les points de la map   //Parcours avec à un iterateur
  Ecrire les coordonnées du point courant dans le fichier .off
  Sauter une ligne dans le fichier .off
  Mettre à jour la valeur correspondant au point courant avec le compteur
  Incrementer le compteur
Fin Pour
\end{lstlisting}


Ensuite, nous parcourons le \textit{vecteur} stockant les point de chaque face pour en écrire
les indices en respectant la structure du fichier \textit{.off.} (voir pseudo code).

\begin{lstlisting}
Pour toutes les faces du vecteur   //Boucle for
  Ecrire le nombre de points de la face courante dans le fichier .off
  Pour tous les points de la face
    Ecrire l'indice correspondant au point courant dans le fichier .off
    Si le point courant n'est pas le dernier de la face
      Ajouter un espace
    Fin Si
  Fin Pour
  Sauter une ligne dans le fichier .off
Fin Pour
\end{lstlisting}

La fonction décrite permet d'obtenir des fichiers qui seront lisibles par un logiciel de
visualisation tel que MeshLab. Ces fichiers sont également conformes à ce qui est
attendu par la classe \textit{Polyhedron}.

D'autre part, nous utilisons une méthode semblable pour écrire les fichiers \textit{.off}
qui permettent d'afficher les triangulations. Ainsi, nous itérons une première fois
sur les sommets pour lister les points. Comme précedemment, nous utilisons une \textit{map}
pour garder en mémoire l'ordre des points via une indexation. Ensuite, nous itérons sur
les faces de la triangulation et nous récupérons les indices de chaque point dans la
\textit{map} créée à cet effet.


Finalement, ces fichiers permettent de visualiser le complexe et son interface.
