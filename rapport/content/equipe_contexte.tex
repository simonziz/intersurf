\chapter{Mise en Contexte}

\section{Présentation de l'équipe}

L'équipe \gls{capsid} rassemble des chercheurs de l'\gls{inria}, du \gls{cnrs}
et de l'Université de Lorraine. L'équipe compte quatre
chercheurs (David Ritchie, Isaure Chauvot de Beauchêne, Marie-Dominique Devignes et
Bernard Maigret), un membre de l'Université de Lorraine (Sabeur Aridhi), un ingénieur
(Antoine Chemardin) ainsi que trois étudiants en thèse.
Les recherches de l'équipe se focalisent sur deux thèmes principaux :
\begin{itemize}
  \item la modélisation des interactions entre les protéines (simulation en dynamique
  moléculaire)
  \item la classification et la fouille de données des interactions et des structures
  protéiques
\end{itemize}
La compréhension du fonctionnement des systèmes biologiques au niveau des structures
moléculaires 3D est au centre des travaux de l'équipe.
Le logiciel Hex, qui permet d'analyser les interactions entre protéines via les fichiers
\textit{.pdb}, a notamment été développé par David Ritchie.

\section{Logiciel Existant}

Le logiciel \gls{intersurf} \cite{intersurf} a été développé au sein de l'\gls{inria}.
Il permet d'extraire la surface d'interface entre deux protéines et fonctionne sous forme
de module pour le logiciel \gls{vmd}. Ce dernier permet d'observer les
structures des protéines afin de les analyser. La prise en main de \gls{vmd} et la compréhension de
la publication concernant \gls{intersurf} a constitué la première partie de mon projet.
Ce logiciel fonctionne de manière statique : il ne permet pas d'observer l'interface
si les protéines subissent des déformations.
L'objectif du projet est de développer un logiciel semblable en utilisant \gls{cgal}
afin d'améliorer les temps de calcul. On cherche également à implémenter ceci de manière à permettre
d'intégrer la composante dynamique. En effet, on veut être capable de calculer l'interface
lorsque les protéines sont déformées pour en observer les différents états.

L'utilité du logiciel existant réside notamment dans le fait qu'il permet d'obtenir
un grand nombre d'informations sur les protéines et leur interface. Nous comparerons ces informations
avec nos résultats de manière à vérifier la validité de l'implémentation.
