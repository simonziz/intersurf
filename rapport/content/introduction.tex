\chapter*{Introduction}
\addcontentsline{toc}{chapter}{Introduction}

\begin{itemize}
  \item Présentation INRIA
  \item Logiciel existant
  \item Objectif : Interface entre protéines / CGAL / Triangulation Delaunay
  \item Utilité : Propriétés des atomes, etc.
\end{itemize}

  L'INRIA (Institut National de Recherche en Informatique et en Automatique) est un
  institut français public de recherche en informatique et en mathématiques. Fondé
  en 1967, l'INRIA compte 2600 collaborateurs rassemblés sur différents sites en
  France dont Nancy. Au sein de ce centre, l'équipe CAPSID développe des algorithmes
  et des logiciels permettant d'étudier des phénomènes et systèmes biologiques
  d'un point de vue structurel, grâce notamment à la modélisation 3D. J'ai effectué
  mon stage au sein de cette équipe sous la tutelle de Dave Ritchie qui en est le
  directeur.

  Supervisé par Dave Ritchie et Bernard Maigret, ce projet a pour objectif de
  modéliser l'interface de contact entre deux protéines. Les propriétés de l'interface
  peuvent en effet donner de nombreuses informations sur les interactions entre
  des protéines. Ceci sert notamment en biologie et en recherche médicinale pour
  le développement de nouveaux médicaments.
  Le projet se déroule également en partenariat avec l'équipe de recherche Vegas, et
  notamment Olivier Devillers, qui a participé à l'imlémentation de
  la librairie d'algorithme de calcul géométrique CGAL (Computational Geometry
  Algorithms Library).

  Cette librairie va permettre, grâce aux nombreuses fonctionnalités qu'elle offre,
  d'améliorer et d'accélerer le développement de la méthode choisie pour le projet.
  Nous détaillerons dans ce rapport la méthode théorique retenue pour approximer l'interface
  entre les protéines : la triangulation de Delaunay et le diagramme de Voronoï. Nous
  expliquerons également comment les structures fournies par CGAL sont utilisées lors
  du développement logiciels en détaillant certaines parties cruciales de l'implémentation.
  Enfin, nous verrons comment les résultats obtenus (et les difficultés rencontrées)
  permettentt d'envisager des perspectives d'avenir pour ce projet.
