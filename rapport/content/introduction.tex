\chapter*{Introduction}
\addcontentsline{toc}{chapter}{Introduction}

\begin{itemize}
  \item Présentation INRIA
  \item Logiciel existant
  \item Objectif : Interface entre protéines / CGAL / Triangulation Delaunay
  \item Utilité : Propriétés des atomes, etc.
\end{itemize}

  L'INRIA (Institut National de Recherche en Informatique et en Automatique) est un
  institut français public de recherche en informatique et en mathématiques. Fondé
  en 1967, l'INRIA compte 2600 collaborateurs rassemblés sur différents sites en
  France dont Nancy. Au sein de ce centre, l'équipe CAPSID développe des algorithmes
  et des logiciels permettant d'étudier des phénomènes et systèmes biologiques
  d'un point de vue structurel, grâce notamment à la modélisation 3D. J'ai effectué
  mon stage au sein de cette équipe sous la tutelle de Dave Ritchie qui en est le
  directeur.

  Supervisé par Dave Ritchie et Bernard Maigret, ce projet a pour objectif de
  modéliser l'interface de contact entre deux protéines en utilisant au mieux
  la librairie d'algorithme de calcul géométrique CGAL (Computational Geometry
  Algorithms Library).
  
